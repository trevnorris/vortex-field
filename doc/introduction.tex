\section*{Author's Note}

I am not a physicist. I am a computer programmer who set out to test the modern capabilities of AI with what was meant to be a weekend experiment. This paper was never supposed to exist.

My initial goal was simple: explore how far AI could push a conceptual physics model before reaching its limits. I had long been fascinated by two historical ideas---Tesla's conception of the aether and Maxwell's vortex model of electromagnetism---and wondered what would happen if these concepts were combined using modern mathematical tools. I created a set of postulates describing particles as vortices in a four-dimensional superfluid, expecting to quickly find contradictions or failures.

Instead, something unexpected happened. With AI assistance in applying the mathematics, the postulates led to field equations. The field equations led to particle mass predictions accurate to fractions of a percent. These led to gravitational phenomena matching general relativity. Each result prompted the next question: ``What else can this explain?''

Throughout this process, I used SymPy to verify every derivation, check dimensional consistency, and ensure mathematical rigor. My goal remained constant: find where the framework breaks. Give it a fair shot, but find its limits. After weeks of testing increasingly complex phenomena---from Mercury's perihelion to binary pulsar decay---the model continued delivering precise results.

This paper represents the accumulated findings of that extended experiment. Every calculation has been symbolically verified. Every prediction has been checked against experimental data. The mathematical patterns that emerged were not designed or expected---they simply appeared from the initial postulates.

I present this work not as a claim to have found ``the answer,'' but as a discovery of remarkable mathematical patterns that demand explanation. The framework makes specific, testable predictions. It reduces dozens of parameters to a handful of geometric inputs. Most importantly, it can be wrong---the 33 GeV four-lepton prediction, the threefold baryon structure, and other novel predictions provide clear tests.

As someone outside academia, I have no career to protect, no theoretical framework to defend, no institutional pressure to conform. My only commitment is to follow the mathematics wherever it leads. If this framework is wrong, I want to know where and why. If it's right, even partially, then perhaps approaching physics from outside the field has allowed fresh perspectives to emerge.

I invite physicists to examine these results critically. Test the predictions. Find the flaws. Verify or refute the mathematics. Science advances through such challenges, and this framework---born from curiosity and computational tools rather than traditional physics training---offers plenty to challenge.

The truth seems to have found me through this unlikely path. Now I offer it to the physics community to determine whether what I've found is profound insight, fortunate coincidence, or instructive error.

\section{Introduction: Unsolved Problems in Fundamental Physics}

Three of physics' deepest mysteries---the origin of particle masses, the weakness of gravity, and quark confinement---have resisted explanation for decades. The Standard Model requires roughly 20 free parameters to describe particle masses and interactions, offering no insight into why the electron weighs 0.511 MeV while the muon weighs 105.66 MeV. General relativity and quantum mechanics remain fundamentally incompatible, with gravity appearing $10^{40}$ times weaker than other forces for reasons unknown. Meanwhile, quantum chromodynamics describes but doesn't explain why quarks can never be isolated, requiring ever-increasing energy to separate them until new particles materialize instead.

What if these seemingly disparate puzzles share a common mathematical structure? We present a framework where particles are modeled as topological defects in a four-dimensional medium, yielding accurate mass predictions, emergent general relativity, and electromagnetism. While the physical interpretation remains open, the mathematical patterns discovered suggest deep geometric and topological principles may underlie particle physics. This paper explores these correspondences without claiming to describe fundamental reality.

% Drop-in replacement subsections for Introduction
% Place these after your opening paragraph about the three mysteries

\subsection{The Mass Hierarchy Problem}

Why does the electron have a mass of precisely 0.511 MeV, the muon 105.66 MeV, and the tau 1776.86 MeV? The Standard Model treats these as free parameters, adjusted to match experiment without explanation. The situation extends across all fermions: six quarks and six leptons with masses spanning twelve orders of magnitude, from the electron neutrino's sub-eV scale to the top quark's 173 GeV. Each mass requires a separate Yukawa coupling constant, hand-tuned, with no predictive framework.

This ad-hoc approach stands in stark contrast to other areas of physics where fundamental principles determine observables. In atomic physics, the Rydberg constant emerges from quantum mechanics and electromagnetism. In thermodynamics, the gas constant follows from statistical mechanics. Yet particle masses---arguably the most basic property of matter---remain mysterious inputs rather than derivable outputs.

Our framework derives lepton masses from geometric structures in four dimensions. The electron, muon, and tau emerge as $n=1,2,3$ quantized vortex configurations, with masses following a golden-ratio scaling pattern. This scaling emerges from energy minimization under a self-similarity symmetry: adding one helical layer, then rescaling by the map $r\mapsto 1+1/r$. The resulting predictions match experiment to $-0.18\%$ (muon) and $+0.10\%$ (tau), reducing the Standard Model's numerous mass parameters to geometric anchors.

For baryons, we propose a fundamentally different structure: a single quantized vortex loop supporting a stable three-lobe standing wave pattern. This naturally explains why `quarks' have never been observed in isolation---they don't exist as separate entities but rather as inseparable phases of a single topological structure. We develop this approach in Sections~\ref{sec:baryons-inside}--\ref{sec:baryons-phenomenology}.

\subsection{The Confinement Puzzle}

Equally mysterious is why the Standard Model's putative constituents of protons and neutrons (quarks) are never isolated---why nature enforces an absolute prohibition on free color charge. In our framework, there are no separable constituents inside baryons: the observed ``three-ness'' arises from a stable three-lobe standing wave on a single closed loop; see Sections~\ref{sec:baryons-inside}--\ref{sec:baryons-phenomenology}.

Our framework suggests confinement isn't a puzzle to solve but a hint that `quarks' don't exist as separate particles. Instead, baryons are single quantized vortex loops with a three-lobe standing wave pattern around their circumference. Attempting to isolate one lobe creates an energy-increasing phase discontinuity---the lobes cannot be separated any more than you can have a wave with only crests and no troughs. This geometric picture naturally explains both the observed three-fold structure of baryons and the impossibility of free quarks.

\subsection{Our Approach}

Rather than adding mathematical complexity to force unification, we explore whether geometric patterns in four dimensions naturally reproduce observed physics. The framework models particles as topological defects---quantized vortices---in a four-dimensional medium, with our three-dimensional universe as a projection surface.

\textbf{What we claim}: The mathematical patterns discovered through this approach match experimental data with remarkable precision, suggesting deep geometric principles underlie particle physics.

\textbf{What we don't claim}: That spacetime ``is'' a superfluid or that particles ``are'' vortices in any ontological sense. These are mathematical tools that reveal constraints any successful theory must satisfy.

The model requires minimal inputs:
\begin{itemize}
\item Two calibrated parameters: Newton's $G$ and speed of light $c$
\item Geometric scales: core size $\xi_c$, circulation quantum $\kappa$
\item No fine-tuning, no landscape of $10^{500}$ vacua
\end{itemize}

From these, the framework derives:
\begin{itemize}
\item Lepton masses matching experiment within $0.2\%$
\item Gravitational phenomena matching GR through 2.5PN order
\item A would-be fourth lepton at 16.48 GeV that cannot form (testable)
\item Baryon structure from single tri-phase closed loops (confinement emerges topologically)
\item Quantum mechanics from vortex phase dynamics
\end{itemize}

\subsection{Key Predictions and Experimental Tests}

The framework makes concrete, falsifiable predictions that distinguish it from the Standard Model:

\paragraph{Already confirmed predictions:}
\begin{itemize}
\item Mercury perihelion advance: $43.0''$/century (observed: $42.98 \pm 0.04$)
\item GP-B frame-dragging: 39 mas/yr (observed: $37.2 \pm 7.2$)
\item Binary pulsar decay: $-2.40\times10^{-12}$ (observed: $-2.423 \pm 0.001\times10^{-12}$)
\item Lepton masses: electron (exact), muon ($-0.18\%$), tau ($+0.10\%$)
\end{itemize}

\paragraph{Near-term testable predictions:}
\begin{itemize}
\item \textbf{4-lepton anomaly}: Excess production near $\sqrt{s} = 33$ GeV without resonance
  \begin{itemize}
  \item No narrow peak at 16.48 GeV (the would-be fourth lepton mass)
  \item Enhanced $\tau^+\tau^-e^+e^-$ over $\mu^+\mu^-\mu^+\mu^-$ (preliminary)
  \item Prompt decay (sub-mm vertices)
  \end{itemize}
\item \textbf{Matter-wave corrections}: $\omega(k) = \frac{\hbar k^2}{2m}\left[1 + \beta_4\frac{k^2}{k_*^2}\right]$ with $k_* \sim \xi^{-1}$
\item \textbf{Intrinsic decoherence}: $\Gamma(d) = \Gamma_0 + \gamma_2 d^2$ scaling with path separation
\end{itemize}

\paragraph{Baryon predictions (Section~\ref{sec:baryons-phenomenology}):}
\begin{itemize}
\item Threefold harmonic in nucleon form factors: $F(q) \sim F_0(q) + F_3(q)\cos(3\varphi)$
\item Correlated changes in mass, magnetic moment, and charge radius for excitations
\item No isolated ``quarks'': apparent quark-like signals are internal tri-lobe phases of a single closed loop
\item Periodic table of baryons indexed by integers $(n_3, k, w, K)$ not constituents
\end{itemize}

\paragraph{What would falsify the model:}
\begin{itemize}
\item Discovery of a stable fourth lepton
\item Narrow resonance at any energy in 4-lepton channels
\item Violation of the golden-ratio mass ladder scaling
\item Absence of threefold harmonics in baryon form factors \emph{in kinematic regimes where the model predicts them}
\item Free quarks observed in any experiment
\end{itemize}

\subsection{Philosophical Stance}

This framework is primarily a tool for discovering mathematical patterns in particle physics. The history of physics shows that mathematical structures often precede physical understanding---complex numbers in quantum mechanics preceded their interpretation as probability amplitudes by decades; Riemannian geometry existed long before Einstein recognized its relevance to gravity. We present our results in this spirit: as precise mathematical correspondences that constrain possible theories.

The word ``aether'' carries historical baggage from failed 19th-century theories, but our approach differs fundamentally from classical aether models. We make no claim of a preferred reference frame for electromagnetic waves, no prediction of aether drag, and all observable phenomena respect Lorentz invariance. The 4D medium, if it exists physically, operates at scales and in dimensions outside direct observation. What matters are the patterns it reveals and the predictions it makes.

Whether nature actually employs vortices in higher dimensions is less important than the fact that this mathematical framework:
\begin{itemize}
\item Reduces dozens of free parameters to a handful of geometric inputs
\item Derives previously unexplained mass ratios
\item Predicts new phenomena at specific energies
\item Unifies disparate physics within a single geometric picture
\end{itemize}

\subsection{Scope and Current Status}

\paragraph{What the framework successfully explains:}
\begin{itemize}
\item Lepton mass hierarchy via golden-ratio scaling from self-similar vortices
\item Absence of a fourth charged lepton (exceeds stability threshold)
\item Baryon confinement as natural consequence of tri-phase loop structure
\item Gravitational phenomena from Newtonian to strong-field regimes
\item Quantum mechanics as emergent from vortex phase dynamics
\item Electromagnetic fields from helical twist projections
\end{itemize}

\paragraph{Novel predictions being tested:}
\begin{itemize}
\item 4-lepton excess at 33 GeV pair-production threshold
\item Threefold structure in baryon form factors
\item High-momentum dispersion in matter-wave interferometry
\item Intrinsic decoherence with characteristic $d^2$ scaling
\end{itemize}

\paragraph{Areas under active development:}
\begin{itemize}
\item Detailed fitting of baryon spectrum using tri-phase model
\item Meson structure (possibly $m=2$ modes on similar loops)
\item Neutrino oscillation parameters from geometric phases
\item CP violation mechanisms
\item Correspondence between integer labels and traditional quantum numbers
\end{itemize}

\paragraph{Reserved for future work:}
\begin{itemize}
\item Dark matter candidates (higher-$n$ vortex states)
\item Dark energy (vacuum configuration of the 4D medium)
\item Cosmological evolution and inflation
\item Strong CP problem
\item Hierarchy between electroweak and Planck scales
\end{itemize}

\subsection{Why This Matters}

If correct, this framework represents a paradigm shift in how we understand particle physics:

\begin{itemize}
\item \textbf{Unification through geometry}: Rather than adding forces and dimensions, all phenomena emerge from vortex dynamics in just one extra dimension
\item \textbf{Parameter reduction}: Dozens of Standard Model parameters reduce to a few geometric inputs
\item \textbf{Conceptual clarity}: Mysterious phenomena like confinement become natural consequences of topology
\item \textbf{Testable predictions}: Specific energies and signatures distinguish this from other approaches
\item \textbf{Mathematical beauty}: The golden ratio and other mathematical constants emerge from physical principles rather than numerology
\end{itemize}

The framework's precision---sub-percent accuracy for masses, exact matches for gravitational tests---using minimal inputs suggests we may be glimpsing fundamental geometric principles that constrain any successful theory of nature.

\subsection{Verification and Reproducibility}
\label{subsec:verification}

Every equation, unit, and derivation in this manuscript was checked by an automated SymPy-based verification suite comprising \textbf{over 2{,}400 tests} (about \textbf{33{,}000 lines of code}). The suite validates:
\begin{itemize}
  \item \textbf{Dimensional consistency} of all expressions (including SI, Gaussian, and Heaviside–Lorentz variants).
  \item \textbf{Conservation laws} (continuity relations and source terms).
  \item \textbf{Poisson and wave equations} with correct prefactors and source dimensions.
  \item \textbf{4D$\to$3D projection identities} and dictionary relations used throughout the text.
  \item \textbf{Asymptotic limits and special cases} to guard against hidden inconsistencies.
\end{itemize}

The full test suite and code are openly available. Readers can clone the repository and run the checks themselves:
\begin{center}
\url{https://github.com/trevnorris/vortex-field}
\end{center}

The verification helper library standardizes symbol dimensions, unit systems, and common checks to ensure consistency across sections of the paper. See the repository for exact instructions to reproduce the results and view detailed summaries of all passes/failures.

\subsection{Related Work}

This model draws inspiration from historical and modern attempts to describe gravity through fluid-like media, but distinguishes itself through its specific 4D superfluid framework and emergent unification in flat space. Early aether theories, such as those discussed by Whittaker in his historical survey \cite{whittaker1951history}, posited a luminiferous medium for light propagation, often conflicting with relativity due to preferred frames and drag effects. In contrast, our approach avoids aether drag by embedding dynamics in a 4D compressible superfluid where perturbations propagate at $v_L$ in the bulk (potentially $>c$) but project to $c$ on the 3D slice with variable effective speeds, preserving Lorentz invariance for observable phenomena through acoustic metrics and vortex stability.

More recent alternatives include Einstein-Aether theory \cite{jacobson2004einstein}, which modifies general relativity by coupling gravity to a dynamical unit timelike vector field, breaking local Lorentz symmetry to introduce preferred frames while recovering GR predictions in limits. Unlike Einstein-Aether, our model remains in flat Euclidean 4D space without curvature, deriving relativistic effects purely from hydrodynamic waves and vortex sinks.

Analog gravity models provide closer parallels, particularly Unruh's sonic black hole analogies \cite{unruh1981experimental}, where fluid flows simulate event horizons and Hawking radiation via density perturbations in moving media. Extensions to superfluids, such as Bose-Einstein condensates \cite{steinhauer2016hawking}, and recent works on vortex dynamics in superfluids mimicking gravitational effects \cite{svancara2024rotating}, demonstrate emergent curved metrics from collective excitations with variable sound speeds. Our framework extends these analogs to a fundamental theory: particles as quantized 4D vortex tori draining into an extra dimension, yielding not just black hole analogs but a full unification of matter and gravity with falsifiable predictions.

A particularly relevant development is the 2024 breakthrough in knot solitons \cite{eto2024knots}, which demonstrated that stable knotted field configurations can indeed serve as particle models---a genuine revival of Lord Kelvin's 1867 vortex atom hypothesis \cite{thomson1867vortex}. This provides modern support for topological approaches to particle physics.

Other geometric unification attempts offer instructive contrasts. String theory requires 10 or 11 dimensions with Calabi-Yau compactifications \cite{candelas1985vacuum}, predicting a landscape of $10^{500}$ possible vacua without selecting our universe. Connes' non-commutative geometry \cite{chamseddine2007gravity} successfully predicted the Higgs mass but provides constraints rather than dynamics. Loop quantum gravity \cite{ashtekar1986new} quantizes spacetime itself but struggles with matter coupling. In each case, mathematical abstraction increases while predictive power for specific observables remains challenging.

Our framework inverts this trend: starting from concrete fluid dynamics in just one extra dimension, it derives specific, testable predictions across particle physics and gravity. The mathematical simplicity---undergraduate-level fluid mechanics rather than advanced differential geometry---makes it accessible while the precision of its predictions demands explanation regardless of one's opinion about the underlying physical picture.
