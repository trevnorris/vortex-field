\section{Electromagnetism from projected circulation}
\label{sec:EM_projection}

We show that the electromagnetic (EM) field on the physical slice $\Pi=\{w=0\}$ arises from the projected kinematics of a 4D aether and its continuity. The \emph{homogeneous} Maxwell equations are kinematic/topological identities of the projected circulation; the \emph{inhomogeneous} pair follow from slice continuity plus a simple linear closure. Physical constants are fixed by the static Coulomb limit and the wave speed $c$, yielding the standard Maxwell system. Throughout we add everyday pictures so a reader can track the physics without following every derivation.

\subsection{Topological charge and projected electromagnetism}
\label{sec:projected-em:charge}

\paragraph{Threading charge in the transition slab.}
Let $\Omega_{\mathrm{TP}}=\Pi\times(-\ell_{\mathrm{TP}}/2,\ell_{\mathrm{TP}}/2)_w$ be the transition-phase slab.
For any large loop $C_R\subset\Omega_{\mathrm{TP}}$ with a spanning surface contained in the slab, define
\begin{equation}
\label{eq:Q-threading}
Q \;:=\; \frac{1}{\kappa}\,\lim_{R\to\infty}\oint_{C_R}\mathbf v\cdot d\boldsymbol\ell.
\end{equation}
% Here \kappa is the circulation quantum of the hidden slab mode; we take
% \kappa = h/m (m: medium's effective mass parameter) so that \Gamma = n\,\kappa.
If the defect core is closed within $\Omega_{\mathrm{TP}}$, then $Q\in\mathbb Z$ and is invariant under smooth deformations of $C_R$ staying inside the slab. If the core intersects and exits the slab (a through-strand), $Q=0$.

\noindent\textbf{Oriented puncture count (equivalent form).}
With the sign convention that +$w$ is ``upward'' and −$w$ is ``downward'' across the slab,
\begin{equation}
Q \;=\; N_{+w}-N_{-w}\,,
\label{eq:Q-oriented-count}
\end{equation}
i.e. $Q$ is the net number of oriented punctures of the slab by the closed core. Neutrality is $Q{=}0$ either as a single $(+w,-w)$ pair or as many micro through--back excursions whose oriented counts cancel.

\paragraph{Neutrino neutrality with drag.}
A neutrino corresponds to a defect whose core does not close within $\Omega_{\mathrm{TP}}$; thus $Q=0$ even though the solenoidal (Eddies) component of the projected flow and the slice-integrated angular momentum (``drag'') can be nonzero. This resolves the ``drag $\Rightarrow$ Eddies $\Rightarrow$ charge'' tension: drag is dynamical, charge is topological.

\paragraph{Coupling strength versus topological charge.}
For through-strands with $Q=0$, projected EM \emph{fields} can still be induced locally in matter via weak polarization/drag couplings. We parametrize their strength—not the charge—by
\begin{equation}
\label{eq:SEM}
S_{\rm EM}(\zeta)\;=\;\exp\!\big[-\beta_{\rm EM}\,\zeta^{\,p}\big],\qquad
\zeta:=\Delta w/\xi_c,\ \ p\in\{2,4\},\ \ \beta_{\rm EM}=O(1\!-\!10),
\end{equation}
which depends on slab overlap. This factor modulates neutral-current–like effects (e.g., tiny polarization or phonon channels) but leaves the binary, topological nature of $Q$ unchanged.

\begin{tcolorbox}[title=Scales and small parameters (at a glance)]
$\xi$ = core radius (microstructure scale), \ \
$\ell_{\mathrm{TP}}$ = slab thickness, \ \
$\ell$ = on-slice feature scale.\\
$\varepsilon_\rho=\xi/\ell$, \ $\varepsilon_\xi=\ell_{\mathrm{TP}}/\ell$, \ $\varepsilon_v=v/c$, \ $\varepsilon_\kappa=\kappa\ell$.
\end{tcolorbox}



\subsection{What the fields are, in math and in pictures}
Let $u(\mathbf x,w,t)$ be the aether velocity in $\mathbb{R}^4$ and $\Omega=\nabla\!\times u$ its (spatial) vorticity. Project onto $\Pi$ and Helmholtz–decompose the induced slice velocity $v(\mathbf x,t)$ as
\[
v(\mathbf x,t)=\nabla\phi(\mathbf x,t)+\nabla\times\mathbf A(\mathbf x,t),
\]
with $\nabla\!\cdot\!\mathbf A=0$ for convenience (Coulomb gauge). We \emph{define} the EM fields by
\begin{equation}
\mathbf B := \nabla\times\mathbf A,
\qquad
\mathbf E := -\,\partial_t \mathbf A \;-\; \nabla \Phi,
\label{eq:EM_defs}
\end{equation}
where $\Phi$ is the slice potential associated with the continuity sector.
% Terminology bridge:
%  - \mathbf A is the Eddies (magnetic) potential; \mathbf B=\nabla\times\mathbf A is the on-slice circulation map.
%  - \Phi is the Slope (Coulomb) potential; -\nabla\Phi is the potential piece of \mathbf E.
%  - -\partial_t \mathbf A is the Induction (loop electric) piece of \mathbf E when Eddies change.

\begin{tcolorbox}[title=Plain-language map]
\textbf{Slope} = the hill/valley (Coulomb) part of $\mathbf E$ from the potential $\Phi$.\\
\textbf{Eddies} = the magnetic field $\mathbf B=\nabla\times\mathbf A$ (on-slice whirls).\\
\textbf{Induction} = the loop electric field $-\partial_t\mathbf A$ that appears when Eddies change in time.\\
\textbf{Displacement-current bridge} = the thin-slab fix that keeps continuity exact across gaps (the ``bulk bridge'').
\end{tcolorbox}

\noindent\emph{Notation.} Where needed (e.g.\ in Sec.~\ref{sec:baryons-inside}), we denote by $\phi_E$ the local phase of the low-order Slope modulation that sets a reference orientation for azimuthal textures on a loop.

\paragraph{Everyday pictures.}
\begin{itemize}
  \item \textbf{Hills and valleys (potential piece).} On $\Pi$ imagine a gentle height map: tiny ``hills'' where the aether is slightly in excess, tiny ``valleys'' where it's slightly depleted. The downhill push is $\mathbf E_{\text{pot}}=-\nabla\Phi$. Positive charge $\Rightarrow$ hilltop; negative charge $\Rightarrow$ valley. Field lines go from hills to valleys. (This is the \textbf{Slope} piece of $\mathbf E$.)
  \item \textbf{Eddies (magnetic/solenoidal piece).} The aether forms on-slice Eddies (whirls with no loose ends); their field map is $\mathbf B=\nabla\times\mathbf A$. When the Eddies pattern \emph{changes in time}, it drives a loop electric field: $\mathbf E_{\text{ind}}=-\partial_t\mathbf A$ (Induction). Faraday's law: changing Eddies $\Rightarrow$ loop electric field (Induction).
  \item \textbf{Charging a capacitor (displacement-current bridge).} Between two plates, some aether briefly ``steps into'' the $w$ direction to keep continuity (the ``bulk bridge''). On the slice this shows up as a time-changing $\mathbf E$ that carries current even through vacuum: the displacement current.
\end{itemize}

\subsection{Two EM laws that are pure kinematics}
By construction,
\begin{equation}
\nabla\!\cdot\!\mathbf B = 0,
\qquad
\nabla\times\mathbf E + \partial_t \mathbf B = 0.
\label{eq:homogeneous}
\end{equation}
These are identities on $\Pi$: divergence of a curl vanishes, and $-\partial_t(\nabla\times\mathbf A)$ cancels the curl of $-\partial_t\mathbf A$.

\paragraph{Everyday pictures.}
\begin{itemize}
  \item \textbf{$\nabla\!\cdot\!\mathbf B=0$:} whirlpools have centers but not endpoints --- like eddies in a river; no ``loose ends'' to source or sink $\mathbf B$.
  \item \textbf{Faraday's law:} wave a magnet near a wire loop and watch the galvanometer wiggle. Changing Eddies $\to$ loop electric field (Induction) $\to$ current.
\end{itemize}

\subsection{Small parameters and regime of validity (EM)}
\label{sec:EM_validity}
We work in the thin–slow–flat slice limit with controlled remainders. Let $\ell$ be the on–slice length scale of interest and $\kappa$ the local curvature of the slice. Define the dimensionless expansion parameters
\[
\varepsilon_\rho := \frac{\xi}{\ell},\qquad
\varepsilon_v := \frac{v}{c},\qquad
\varepsilon_\xi := \frac{\ell_{\mathrm{TP}}}{\ell},\qquad
\varepsilon_\kappa := \kappa\,\ell.
\]
\emph{Notation note:} here $\varepsilon_\rho$ denotes the geometric smallness $\xi/\ell$. When a density-contrast amplitude is needed in wave-sector derivations, we write it explicitly as $\delta\rho/\rho_0$.

All field equations in this section hold to leading order in these small numbers; we indicate remainders schematically as $O(\varepsilon_\rho^2+\varepsilon_v^2+\varepsilon_\xi^2+\varepsilon_\kappa^2)$. For conventions and units (signature, index placement, SI vs. natural units), see Sec.~\ref{sec:motivation-conventions}.

\subsection{Where sources come from: continuity and the displacement-current bridge}
Let $\rho(\mathbf x,t)$ be the projected aether excess density on $\Pi$ and $\mathbf J(\mathbf x,t)$ the in-slice transport current. A thin pillbox straddling $\Pi$ turns 4D continuity into
\begin{equation}
\partial_t \rho + \nabla\!\cdot\!\mathbf J \;=\; -\,\Big[J_w\Big]_{w=0^-}^{0^+},
\label{eq:slice_continuity}
\end{equation}
with $J_w$ the normal flux into/out of the bulk. We close the potential sector by the minimal linear, local response
\begin{equation}
-\nabla^2 \Phi = \frac{\rho}{\varepsilon_0},
\qquad
\text{so that}\quad
\nabla\!\cdot\!\big(\partial_t\mathbf E_{\text{pot}}\big)=\frac{1}{\varepsilon_0}\,\partial_t\rho,
\label{eq:closure}
\end{equation}
and we \emph{identify} the normal flux with the displacement current supplied by the time-varying potential sector,
\begin{equation}
\Big[J_w\Big]_{w=0^-}^{0^+} \;=\; -\,\varepsilon_0\,\nabla\!\cdot\!\partial_t\mathbf E_{\text{pot}}.
\label{eq:displacement_identification}
\end{equation}
Combining \eqref{eq:EM_defs}–\eqref{eq:displacement_identification} yields the inhomogeneous pair
\begin{equation}
\nabla\!\cdot\!\mathbf E = \frac{\rho}{\varepsilon_0},
\qquad
\nabla\times \mathbf B - \mu_0 \varepsilon_0\,\partial_t \mathbf E = \mu_0\,\mathbf J.
\label{eq:inhomogeneous}
\end{equation}
% Reading with our terms:
%  - \nabla\!\cdot\!\mathbf E = \rho/\varepsilon_0  (Slope/Coulomb piece),
%  - \nabla\times\mathbf B - \mu_0\varepsilon_0\,\partial_t\mathbf E = \mu_0\,\mathbf J  (Eddies + Induction + bridge).

\paragraph{Everyday pictures.}
\begin{itemize}
  \item \textbf{Gauss's law: hills/valleys make arrows.} Pile a bit of aether on the slice (a hill) and the downhill arrows $\mathbf E$ point outward; scoop some out (a valley) and arrows point inward.
  \item \textbf{Amp\`ere–Maxwell: current or changing hill-tilt makes Eddies.} Push a steady stream along the slice ($\mathbf J$) and you wind $\mathbf B$ around it; tilt the height map in time ($\partial_t\mathbf E$ between capacitor plates) and you wind $\mathbf B$ the same way --- the displacement-current bridge guarantees there is no break in the circuit.
\end{itemize}

\subsection{Fixing the constants and waves}
Taking the curl of Amp\`ere–Maxwell and using \eqref{eq:homogeneous} gives vacuum waves
\[
\big(\nabla^2 - \tfrac{1}{c^2}\partial_{tt}\big)\mathbf E=0,
\qquad
\big(\nabla^2 - \tfrac{1}{c^2}\partial_{tt}\big)\mathbf B=0,
\]
provided
\begin{equation}
c^2=\frac{1}{\mu_0\varepsilon_0}.
\label{eq:c_relation}
\end{equation}
We take $c$ to be the measured wave speed (light in vacuum), which fixes the product $\mu_0\varepsilon_0$. The static Coulomb limit of \eqref{eq:closure} fixes $\varepsilon_0$; then $\mu_0$ follows from \eqref{eq:c_relation}.

\paragraph{Everyday picture.}
\textbf{Ripples on a stretched sheet.} The sheet tension sets the wave speed; here the combination $\mu_0\varepsilon_0$ sets $c$. Once you know $c$ and the static push between charges (Coulomb), all constants are pinned.

\subsection{Minimal Coupling, EM Stress--Energy, and Light Propagation}
\label{sec:EM-coupling}
% SI units in this section; we use \mu_0,\,\varepsilon_0 with c^2 = 1/(\mu_0\varepsilon_0).
All non-gravitational fields couple universally to $g_{\mu\nu}$ via $\eta\to g$.
For electromagnetism,
\begin{equation}
S_{\rm EM}[g,A]=-\frac{1}{4\mu_0}\int d^4x\,\sqrt{-g}\,F_{\mu\nu}F^{\mu\nu},\qquad
\nabla_\mu F^{\mu\nu}=\mu_0 J^\nu,\qquad \nabla_{[\alpha}F_{\beta\gamma]}=0.
\label{eq:EM-covariant}
\end{equation}
The stress--energy is
\begin{equation}
T^{\mu\nu}_{\rm EM}=\frac{1}{\mu_0}\Big(F^{\mu\alpha}F^\nu{}_\alpha-\frac{1}{4}g^{\mu\nu}F_{\alpha\beta}F^{\alpha\beta}\Big),
\label{eq:EM-Tmunu}
\end{equation}
and contributes to the total source in Einstein’s equation.
Light rays follow null geodesics of $g_{\mu\nu}$: $k^\mu k_\mu=0$ and $k^\nu\nabla_\nu k^\mu=0$.
With the strong-field dictionary of Sec.~\ref{sec:strong-dict}, a static spherical vacuum ($\mathbf A_g=0$) yields Schwarzschild in isotropic coordinates and reproduces standard light deflection and Shapiro delay.

\subsection{Energy flow (Poynting theorem), told like a story}
Dot $\mathbf E$ into Amp\`ere–Maxwell, dot $\mathbf B$ into Faraday, subtract, and rearrange:
\[
\partial_t \Big(\tfrac{\varepsilon_0}{2}\,|\mathbf E|^2 + \tfrac{1}{2\mu_0}\,|\mathbf B|^2\Big)
+ \nabla\!\cdot\!\Big(\tfrac{1}{\mu_0}\,\mathbf E\times \mathbf B\Big)
= -\,\mathbf J\!\cdot\!\mathbf E.
\]
\textbf{Conveyor-belt picture:} the crossed fields $\mathbf E\times\mathbf B$ are a belt carrying energy through space; the belt unloads onto charges at rate $\mathbf J\!\cdot\!\mathbf E$.

\paragraph{Brief note on media.}
In linear, isotropic media obtained by coarse-graining microstructure on the slice, the macroscopic fields satisfy $\mathbf D=\varepsilon\,\mathbf E$ and $\mathbf H=\mathbf B/\mu$ with constitutive parameters $(\varepsilon,\mu)$ fixed by the local microstate. The derivations mirror the vacuum case with polarization/magnetization currents included in $J^\mu$. A full treatment is deferred; here we restrict to vacuum ($\varepsilon=\varepsilon_0$, $\mu=\mu_0$).

\subsection{Thickness and accuracy}
If the 4D transition band is smooth, even, and thin of width $\xi$ in $w$, replacing the sharp projection by a convolution changes the induced fields by
\[
\Delta(\cdot)=O\!\big((\xi/\ell)^2\big)
\]
when probed on length $\ell$ on $\Pi$ (second-moment Taylor estimate), matching the curvature/thickness control used elsewhere. This is why the textbook Maxwell theory works so well over a vast range: corrections are quadratically suppressed by the small ratio $\xi/\ell$.

\subsection{Beyond-Maxwell predictions and falsifiable tests}
\label{subsec:EM_predictions}
The homogeneous laws \eqref{eq:homogeneous} are exact (topology). Any deviation must come from the \emph{closure} of the potential/continuity sector. A smooth, even transition profile of width $\xi$ and (optionally) a finite bulk-exchange time $\tau$ give the following leading, \emph{scale-suppressed} effects. Each comes with a clean scaling law, so null results set direct bounds on $\xi$ and $\tau$.

\paragraph{A. Static near-field: tiny universal Coulomb correction.}
A minimal local closure augments Poisson by the next even derivative:
\begin{equation}
\big(-\nabla^2 + \alpha\,\xi^2 \nabla^4 + \cdots\big)\,\Phi
=\frac{\rho}{\varepsilon_0},\qquad \alpha=O(1).
\end{equation}
For a point charge,
\begin{equation}
\Phi(r)=\frac{q}{4\pi\varepsilon_0 r}
\Big[1-\alpha\,\tfrac{\xi^2}{2r^2}+O\big((\xi/r)^4\big)\Big],
\quad
\Rightarrow\quad
|\mathbf E|=\frac{q}{4\pi\varepsilon_0 r^2}\Big[1-\alpha\,\tfrac{3\xi^2}{2r^2}+\cdots\Big].
\end{equation}
\emph{Test:} precision force/field measurements in ultra-clean nanogaps (AFM/STM-style). A null at fractional precision $\delta$ at gap $r$ implies $\xi \lesssim r\sqrt{\delta}$.

\paragraph{B. Vacuum wave dispersion at very high frequency.}
Finite thickness yields the first isotropic, Lorentz-breaking correction
\begin{equation}
\omega^2=c^2 k^2\Big[1+\sigma\,(k\xi)^2+O\big((k\xi)^4\big)\Big],\qquad \sigma=O(1),
\end{equation}
so the group velocity $v_g\simeq c\big[1+\tfrac{3}{2}\sigma\,(k\xi)^2\big]$.
\emph{Test:} dual-color ultra-stable optical cavities or femto/atto-second time-of-flight over meter-scale vacuum paths; look for a $\propto\lambda^{-2}$ shift. Null $\Rightarrow$ bound on $\xi$ (and $\sigma$).

\paragraph{C. Ultrafast transients: even-in-time displacement memory.}
A causal, non-dissipative bulk exchange gives
\begin{equation}
\varepsilon(\omega)=\varepsilon_0\Big[1+\beta\,(\omega\tau)^2+O\big((\omega\tau)^4\big)\Big],\qquad \beta=O(1),
\end{equation}
equivalently a $\tau^2\partial_{tt}\mathbf E$ correction in time domain.
\emph{Test:} THz time-domain spectroscopy of ultrafast parallel-plate nanocapacitors; fit phase curvature $\propto(\omega\tau)^2$ (even in $\omega$). Null $\Rightarrow$ bound on $\tau$.

\paragraph{D. Nanoscale boundaries: universal cavity mode shifts.}
Effective boundary conditions pick up an $O(\xi)$ slip in tight confinement (transverse scale $a$), giving
\begin{equation}
\frac{\Delta f}{f}=+\gamma\Big(\frac{\xi}{a}\Big)^2 + O\big((\xi/a)^4\big),\qquad \gamma=O(1),
\end{equation}
independent of polarization at this order.
\emph{Test:} compare families of high-$Q$ dielectric or photonic-crystal nanocavities as $a$ is scaled; look for the quadratic trend after subtracting known systematics.

\paragraph{E. Strong-field nonlinearity with a definite sign.}
Field energy slightly perturbs aether density, feeding back into the closure and producing a Kerr-like index
\begin{equation}
n(I)\simeq 1+n_2 I,\qquad n_2>0 \ \ \text{(sign fixed by positive compressibility)}.
\end{equation}
\emph{Test:} high-finesse cavity self-phase modulation in ultra-high vacuum using multi-GW/cm$^2$ pulses. Compare against the tiny QED Heisenberg–Euler baseline; here the leading symmetry matches (no birefringence at this order) but the \emph{sign} is fixed and the magnitude scales with $\xi,\tau$.

\paragraph{Reading the scalings.}
A single small spatial scale $\xi$ and (optionally) a small temporal scale $\tau$ control all departures: statics $\propto(\xi/r)^2$, dispersion $\propto(k\xi)^2$, confinement $\propto(\xi/a)^2$, ultrafast memory $\propto(\omega\tau)^2$, and a weak, fixed-sign nonlinearity. Multiple nulls across these orthogonal handles rapidly squeeze $(\xi,\tau)$, or a positive signal would over-constrain the same pair.

\subsubsection{Existing bounds (EM-only)}
\label{subsec:EM_existing_bounds}
We summarize how off-the-shelf laboratory precisions already constrain the two transition parameters $(\xi,\tau)$, using the leading scalings stated above. Coefficients $\alpha,\sigma,\beta,\gamma$ are $O(1)$; bounds are shown explicitly with their coefficient dependence.

\begin{itemize}
  \item \textbf{Near-field Coulomb tests (A).} A fractional null $\delta_{\rm C}$ on the $1/r^2$ law at a plate/AFM gap $r$ implies
  \[
  \xi \;\lesssim\; r\,\sqrt{\delta_{\rm C}/|\alpha|}.
  \]
  Illustrative numbers: $r\sim 100\,\mathrm{nm}$ with $\delta_{\rm C}\sim10^{-2}$ would give $\xi\lesssim10\,\mathrm{nm}$.

  \item \textbf{Vacuum dispersion from dual-color/TOF (B).} A bound $|\Delta v|/v\le \delta_{\rm D}$ between two nearby optical wavelengths (wavenumber $k=2\pi/\lambda$) yields
  \[
  \xi \;\lesssim\; \frac{1}{k}\,\sqrt{\frac{\delta_{\rm D}}{c_\sigma\,|\sigma|}}
  \;=\; \frac{\lambda}{2\pi}\,\sqrt{\frac{\delta_{\rm D}}{c_\sigma\,|\sigma|}},
  \]
  where $c_\sigma$ is an order-unity shape factor set by the precise dispersion expansion used upstream. For example, $\delta_{\rm D}\sim10^{-11}$ at $\lambda\sim1\,\mu\mathrm{m}$ would correspond (formally) to a sub-picometer $\xi$; actual lab systematics decide how tight one trusts this channel.

  \item \textbf{Ultrafast displacement memory (C).} A curvature-in-frequency bound $|\varepsilon(\omega)-\varepsilon_0|/\varepsilon_0\le\delta_\varepsilon$ at angular frequency $\omega$ gives
  \[
  \tau \;\lesssim\; \frac{1}{\omega}\,\sqrt{\delta_\varepsilon/|\beta|}.
  \]
  Example: at THz ($\omega\!\sim\!2\pi\,\mathrm{THz}$) with $\delta_\varepsilon\sim10^{-3}$ one finds $\tau\lesssim\text{a few }\mathrm{fs}$.

  \item \textbf{Nanophotonic cavity shifts (D).} A polarization-agnostic fractional mode shift null $|\Delta f|/f\le\delta_f$ at transverse scale $a$ implies
  \[
  \xi \;\lesssim\; a\,\sqrt{\delta_f/|\gamma|}.
  \]
  Example: $\delta_f\sim10^{-5}$ at $a\sim200\,\mathrm{nm}$ would give $\xi\lesssim\text{a few }\mathrm{nm}$.

  \item \textbf{Strong-field nonlinearity (E).} Existing vacuum Kerr searches constrain $n_2$; in this framework that bound maps to a constraint on the (model-dependent) combination of $(\xi,\tau)$ that controls the closure’s intensity dependence. A dedicated derivation is deferred; we note only that present limits already make this channel subdominant to (A)–(D) in the thin/fast regime.
\end{itemize}

\paragraph{Takeaway.} Even with conservative laboratory precisions, channels (A), (C), and (D) already prefer a very thin and fast transition: $\xi$ well below the on-slice probes used (often \textit{nm}–scale or tighter), and $\tau$ in the \textit{fs} regime or below. Channel (B) can be extremely constraining if one trusts cavity/time-of-flight systematics at the quoted levels; we keep it as a sensitivity lever rather than a hard claim.

\paragraph{Bottom line.}
Maxwell's equations emerge cleanly on the slice; if Nature implements the projection through a perfectly sharp interface, $\xi,\tau\!\to\!0$ and no deviations appear. If the transition is merely very thin/fast, the tests above bound $(\xi,\tau)$ directly.

\subsection{Electromagnetism as Slope+Eddies from Oriented Links}
\label{sec:EM-from-helical-vortices}

% ------------------------------------------------------------
% 30-second story / big picture
% ------------------------------------------------------------
\begin{tcolorbox}[title=Plain-language snapshot (30 seconds)]
Oriented links through the slice write a standing Slope pattern (electric), while motion/Drag organize Eddies (magnetic). Changing Eddies create loop electric fields (Induction). The displacement-current bridge lives in the thin slab to keep continuity exact during transients.\footnote{We remain agnostic about ontology; the subsection only establishes the mathematical map.}
\end{tcolorbox}

\begin{tcolorbox}[title=Terminology note]
``Sheet'' here refers to the \emph{topological world-volume} used to derive Maxwell sources (a $(4{+}1)$D bookkeeping device for oriented links).
\end{tcolorbox}

% ------------------------------------------------------------
% quick dictionary
% ------------------------------------------------------------
\begin{tcolorbox}[title=Quick dictionary (objects \(\leftrightarrow\) meanings)]
\begin{itemize}
  \item Bulk 2-form \(B_{MN}=-B_{NM}\) with fieldstrength \(H_{MNP}=\partial_{[M}B_{NP]}\) $\leftrightarrow$ parent gauge field.
  \item Vector potential \(A_\mu\equiv B_{\mu 4}\) $\leftrightarrow$ electromagnetic potential in $(3{+}1)$D.
  \item Maxwell tensor \(F_{\mu\nu}\equiv H_{\mu\nu 4}=\partial_\mu A_\nu-\partial_\nu A_\mu\).
  \item Sheet world-volume current \(J^{MNP}\) (conserved 3-form) $\leftrightarrow$ topological data of vortex sheets.
  \item Sheet endpoint in $(3{+}1)$D $\leftrightarrow$ point charge of integer strength \(n\) (sign set by orientation).
  \item Sheet endpoint worldline $\leftrightarrow$ electric 4-current \(j_e^\mu\).
  \item Oriented link $\leftrightarrow$ integer charge (sign by orientation), writes the Slope profile.
  \item On-slice Eddies $\leftrightarrow$ magnetic field $\mathbf B$; changing Eddies $\leftrightarrow$ Induction (loop $\mathbf E$).
\end{itemize}
\end{tcolorbox}

\subsubsection{Setup and notation}
We use coordinates $x^M$ with $M=0,1,2,3,4$, where $x^0\equiv t$ and the extra spatial coordinate $x^4$ is compact with circumference $L_4$. Greek indices $\mu,\nu=0,1,2,3$ refer to the observed $(3{+}1)$D slice. We adopt metric signature $(-,+,+,+)$ for statements in $(3{+}1)$D and standard index symmetrization/antisymmetrization conventions. Outside vortex cores, fields are taken $x^4$-independent.

\paragraph*{Idea.} Keep the $x^4$-component of the bulk two-form; it behaves like an electromagnetic vector potential.
\paragraph*{Result.} Define $A_\mu\equiv B_{\mu 4}$ and $F_{\mu\nu}\equiv H_{\mu\nu 4}$ to obtain the usual Maxwell tensor on the slice.

\subsubsection{Topological sheet current and identities}
Vortex \emph{sheets} are codimension-2 defects whose world-volumes in $(4{+}1)$D are captured by a conserved 3-form current $J^{MNP}$,
\begin{equation}
\partial_M J^{MNP}=0,
\qquad
J^{MNP}=-J^{NMP}=\cdots
\end{equation}
The parent gauge field is a Kalb--Ramond 2-form $B_{MN}$ with fieldstrength $H=\mathrm d B$, so
\begin{equation}
H_{MNP}=\partial_{[M}B_{NP]},
\qquad
\partial_{[M}H_{NPQ]}=0\quad\text{(Bianchi)}.
\end{equation}
These identities imply ``no magnetic monopoles'' after dimensional reduction: $\mathrm d F=0$ gives $\nabla\!\cdot\!\mathbf B=0$ and Faraday's law.

\subsubsection{Dimensional reduction and identification of $A_\mu$}
Compactify $x^4\sim x^4+L_4$ and assume $\partial_4$-independence away from cores. Decompose
\begin{equation}
B_{MN}\;\longrightarrow\;\{\,B_{\mu\nu}\,,\;B_{\mu 4}\equiv A_\mu\,\},
\qquad
F_{\mu\nu}\equiv H_{\mu\nu 4}
=\partial_\mu A_\nu-\partial_\nu A_\mu.
\end{equation}
The remaining components $B_{\mu\nu}$ are heavy/decoupled in this sector (or dual to a scalar not sourced by endpoints), so we drop them here. The minimal bulk dynamics and coupling can be summarized schematically as
\begin{equation}
\partial_M H^{MNP}=g_B^{2}\,J^{NP},\qquad \text{with}\quad J^{NP}\equiv J^{NP4} \;\text{(sheet data)}.
\end{equation}
Integrating over the compact direction yields Maxwell's equations on the slice,
\begin{equation}
\partial_\mu F^{\mu\nu}=g_{\rm eff}^2\, j_e^{\,\nu},
\qquad
j_e^{\,\nu}(x)\equiv \int_0^{L_4}\!\mathrm d x^4\; J^{\nu 4}(x,x^4),
\label{eq:maxwell-from-sheets}
\end{equation}
with an effective coupling $g_{\rm eff}^2\propto g_B^2/L_4$ fixed by normalization below.

\begin{tcolorbox}[title=Idea $\to$ Result]
\textbf{Idea:} The $x^4$-component of the topological sheet current looks like an electric current when viewed from $(3{+}1)$D.\\
\textbf{Result:} After integrating over $x^4$, Eq.~\eqref{eq:maxwell-from-sheets} is precisely the sourced Maxwell equation.
\end{tcolorbox}

\subsubsection{Charge quantization from helical twist (Gauss law)}
Let $\mathcal S^2$ be a 2-sphere in the observed space that encloses a single sheet endpoint of integer twist $n$ (orientation convention stated below). The flux of $\,{}^\star\!F\,$ through $\mathcal S^2$ counts the twist:
\begin{equation}
\int_{\mathcal S^2}\!{}^\star F\;=\; q_0\,n,\qquad q_0\equiv \frac{2\pi L_4}{g_B^{2}}\;\;\times\;\,(\text{unit\;choice}).
\label{eq:gauss-quantization}
\end{equation}
Thus electric charge is \emph{topological}: it depends only on the integer $n$, not on core details. We fix the overall sign by the orientation convention: a right-handed helical advance of the sheet along $+x^4$ at the endpoint corresponds to $n>0$ and \emph{positive} charge.

\begin{tcolorbox}
\textbf{Orientation convention.} ``Positive linking'' (right-handed twist advancing toward $+x^4$) $\Rightarrow$ $n>0$ $\Rightarrow$ positive charge in $(3{+}1)$D.
\end{tcolorbox}

\subsubsection{Normalization to standard units}
It is useful to present two equivalent normalizations.
\begin{enumerate}
  \item \textbf{Heaviside--Lorentz (natural) units.} Choose the $(3{+}1)$D action
  \begin{equation}
  S_{\rm EM}=\int\!\mathrm d^4x\;\Big[-\tfrac{1}{4}\,F_{\mu\nu}F^{\mu\nu}+A_\mu j_e^{\,\mu}\Big],
  \end{equation}
  which is achieved by the rescaling $A_\mu\mapsto A_\mu/e$ with
  \begin{equation}
  e^{2}=\frac{g_B^{2}}{2L_4} \quad (\text{up to an overall sign fixed by \eqref{eq:gauss-quantization}}).
  \end{equation}
  Then a single endpoint with twist $n$ carries physical charge $Q= n\,q_0/e$.

  \item \textbf{SI units.} Write $S_{\rm EM}=\int\!\mathrm d^4x\;\big[\tfrac{\epsilon_0}{2}\,\mathbf E^2-\tfrac{1}{2\mu_0}\,\mathbf B^2 + A_\mu j_e^{\,\mu}\big]$ and rescale so that $\epsilon_0\mu_0=1/c^2$. The same mapping gives $e^2= g_B^{2}/(2L_4)$ while the physical charge unit is $q_0/e$; numerical values are then fixed by the chosen calibration (e.g., set $n=-1$ to reproduce the electron charge).
\end{enumerate}
Either choice is fine provided one sticks to it consistently.

\subsubsection{Worked examples (outside cores)}
\begin{enumerate}
  \item \textbf{Single endpoint ($n=1$).} Spherical symmetry gives $|\mathbf E|= (q_0/e)/(4\pi r^2)$ (SI) or $|\mathbf E|=(q_0/e)/r^2$ (Heaviside--Lorentz). The magnetic field vanishes for static endpoints.
  \item \textbf{Dipole ($n=+1$ and $n=-1$ separated by $\mathbf d$).} Far field is the standard dipole: $\mathbf E\approx \tfrac{q_0}{e}\,\tfrac{3(\hat{\mathbf r}\!\cdot\!\mathbf p)\hat{\mathbf r}-\mathbf p}{4\pi r^3}$ with $\mathbf p=\tfrac{q_0}{e}\,\mathbf d$ (SI). Time-dependent motion radiates exactly as in Maxwell theory.
\end{enumerate}

\subsubsection{FAQ (math-only claims)}
\begin{itemize}
  \item \textbf{Why is charge conserved?} Gauge invariance of the parent two-form enforces $\partial_M J^{MNP}=0$, which reduces to $\partial_\mu j_e^{\,\mu}=0$ after integrating over $x^4$.
  \item \textbf{Why no magnetic monopoles?} The Bianchi identity $\mathrm dH=0$ implies $\mathrm dF=0$, hence $\nabla\!\cdot\!\mathbf B=0$ and Faraday's law.
  \item \textbf{What sets the size of the charge quantum?} Geometry and coupling: $q_0=2\pi L_4/g_B^{2}$ (then rescaled by $e$ in the chosen unit system). Integer $n$ labels the helical linking number.
  \item \textbf{Does core microphysics matter?} Not for charge: Eq.~\eqref{eq:gauss-quantization} is topological. Core details enter only in short-distance regularization, not in the flux quantization.
\end{itemize}

\subsubsection{Numerical recipe (practical use)}
\begin{enumerate}
  \item Specify endpoint worldlines $\{x_a^\mu(\tau)\}$ and integers $n_a$ (sign by orientation).
  \item Build $j_e^{\,\mu}(x)=\sum_a (n_a q_0/e)\int \!\mathrm d\tau\; \dot x_a^{\,\mu}(\tau)\,\delta^{(4)}\big(x-x_a(\tau)\big)$.
  \item Solve $\partial_\mu F^{\mu\nu}= j_e^{\,\nu}$ and $\partial_{[\alpha}F_{\beta\gamma]}=0$ with retarded Green's functions (Lorenz gauge is convenient): $A^\mu(x)=\int \mathrm d^4x'\, G_{\rm ret}(x-x')\, j_e^{\,\mu}(x')$.
  \item Report fields in your preferred units; outside cores they are identical to standard Maxwell solutions. Accelerating endpoints (or time-dependent twists) radiate.
\end{enumerate}

% ------------------------------------------------------------
% gentle guardrails / reminders
% ------------------------------------------------------------
\begin{tcolorbox}
\textbf{Conventions used here.} (i) $[\Gamma]=L^2/T$ for circulation; (ii) We keep EM and gravitomagnetic sectors strictly separate (no overloading of $A_\mu$); (iii) Orientation: right-handed twist toward $+x^4$ is positive $n$.
\end{tcolorbox}


\subsection{Electromagnetism in the Wave sector}
\label{subsec:em_wave}

\paragraph{Scope and units.}
In this subsection we isolate the electromagnetic (EM) wave sector. We work in \emph{Gaussian--cgs} units and use the standard 4-potential
\(A^\mu=(\Phi_{\text{Slope}}/c,\mathbf A)\), field tensor \(F_{\mu\nu}=\partial_\mu A_\nu-\partial_\nu A_\mu\),
and physical fields \(\mathbf E=-\nabla\Phi_{\text{Slope}}-\tfrac1c\,\partial_t\mathbf A\), \(\mathbf B=\nabla\times\mathbf A\).
The EM sources are \(J^\mu_{\rm ch}=(c\,\rho_{\rm ch},\,\mathbf J_{\rm ch})\) and obey the
charge continuity equation \(\partial_t\rho_{\rm ch}+\nabla\!\cdot\!\mathbf J_{\rm ch}=0\).
\emph{These EM sources and fields are distinct from any gravitational/``GEM'' sector variables, which do not appear here.}
For conventions and units (signature, index placement), see Sec.~\ref{sec:motivation-conventions}.

Throughout this subsection we write $\mathbf A \equiv \mathbf A_{\text{EM}}$.

We decompose $\mathbf E = -\nabla \Phi_{\text{Slope}} - \tfrac{1}{c}\,\partial_t \mathbf A$ where $\mathbf B=\nabla\times\mathbf A$ (Eddies). The first term is Slope (Coulomb); the second is Induction (loop electric field).

\paragraph{30-second story (reader guide).}
This is the \emph{wave} part of the theory: once the EM field content is defined,
time-dependent sources launch disturbances that propagate causally with speed \(c\).
The static limit reproduces Coulomb's law; the time-dependent regime gives retarded fields and radiative energy flux.

\subsubsection{Field equations, gauge, and radiation condition}
We adopt \emph{Lorenz gauge} \(\partial_\mu A^\mu=0\), which is Lorentz-invariant.
Maxwell's equations in terms of \(F_{\mu\nu}\) are
\begin{align}
\partial_\mu F^{\mu\nu} &= \frac{4\pi}{c}\,J^\nu_{\rm ch}, &
\partial_{[\alpha}F_{\beta\gamma]} &= 0,
\end{align}
and, equivalently, the potentials satisfy the manifestly hyperbolic system
\begin{equation}
\left(\frac{1}{c^2}\partial_t^2-\nabla^2\right) A^\mu
= -\,\frac{4\pi}{c}\,J^\mu_{\rm ch}, \qquad \partial_\mu A^\mu=0,
\label{eq:wave_A_Lorenz}
\end{equation}
valid to leading order in the small parameters of this framework
\((\varepsilon_\rho,\varepsilon_v^2,\varepsilon_\xi)\) (see Sec.~\ref{sec:EM_validity} and Sec.~\ref{sec:motivation-conventions}).
At spatial infinity we impose the \emph{Sommerfeld radiation condition} (outgoing waves only):
\begin{equation}
\left(\partial_r - \frac{1}{c}\partial_t\right)\!\big(r\,A^\mu\big) \;\to\; 0
\quad \text{as}\quad r\to\infty,
\end{equation}
which selects retarded (causal) solutions and ensures a well-defined energy flux.

\noindent\emph{Idea:} assume Lorentz covariance, gauge invariance, and linearity for observables. \;
\emph{Result:} the electromagnetic wave equations \eqref{eq:wave_A_Lorenz} with characteristic speed \(c\).

\subsubsection{Static calibration and Coulomb limit}
In the static limit (\(\partial_t=0\), localized sources), \eqref{eq:wave_A_Lorenz} reduces to
\begin{align}
\nabla^2 \Phi_{\text{Slope}} &= -\,4\pi\,\rho_{\rm ch}, &
\nabla^2 \mathbf A &= \mathbf 0 \quad (\text{Lorenz gauge}).
\end{align}
For a point charge \(Q\) at the origin,
\(\Phi_{\text{Slope}}(\mathbf r)= Q/r\), \(\mathbf E(\mathbf r)= Q\,\mathbf r/r^3\), \(\mathbf B=\mathbf 0\).
This \emph{calibrates} the normalization so that all dynamic results reduce to the familiar Coulomb law when \(\partial_t\to0\).

\noindent\emph{Idea:} require the static sector to match measured electrostatics. \;
\emph{Result:} fix couplings so \(\nabla\!\cdot\!\mathbf E=4\pi\rho_{\rm ch}\) and \(E=Q/r^2\).

\subsubsection{Retarded solutions and causality}
With the radiation condition, the unique solutions of \eqref{eq:wave_A_Lorenz} are the retarded potentials
\begin{align}
\Phi_{\text{Slope}}(\mathbf x,t) &= \int \frac{\rho_{\rm ch}(\mathbf x',\,t_r)}{|\mathbf x-\mathbf x'|}\,d^3x',\\
\mathbf A(\mathbf x,t) &= \frac{1}{c}\int \frac{\mathbf J_{\rm ch}(\mathbf x',\,t_r)}{|\mathbf x-\mathbf x'|}\,d^3x',
\end{align}
with \emph{retarded time} \(t_r=t-|\mathbf x-\mathbf x'|/c\).
Equivalently, one may use the retarded Green's function formalism (see Appendix on retarded Green's functions).

\noindent\emph{Idea:} allow only outbound disturbances consistent with the light cone. \;
\emph{Result:} fields at \((\mathbf x,t)\) depend only on past sources inside the backward light cone; no superluminal signaling.

\subsubsection{Energy density, Poynting flux, and radiation}
The EM energy density and flux in Gaussian units are
\begin{equation}
u = \frac{E^2+B^2}{8\pi}, \qquad \mathbf S = \frac{c}{4\pi}\,\mathbf E\times\mathbf B,
\end{equation}
and Poynting's theorem holds:
\begin{equation}
\partial_t u + \nabla\!\cdot\!\mathbf S = -\,\mathbf J_{\rm ch}\!\cdot\!\mathbf E.
\end{equation}
Under the radiation condition, the total power radiated is the surface integral of \(\mathbf S\cdot d\mathbf a\) over a sphere at large \(r\).

\paragraph{Projected EM self-energy (used in Sec.~\ref{sec:baryons-phenomenology}).}
For any localized, quasi-static configuration, we define
\begin{equation}
\delta E_{\rm EM}[\mathbf E,\mathbf B] \;:=\; \int_{\Pi}\! d^3 x\;\,
\begin{cases}
\tfrac{1}{2}\big(\varepsilon_0 |\mathbf E|^2 + \tfrac{1}{\mu_0}|\mathbf B|^2\big), & \text{(SI)},\\[2pt]
\tfrac{1}{8\pi}\big(|\mathbf E|^2+|\mathbf B|^2\big), & \text{(Gaussian)}.
\end{cases}
\label{eq:deltaE_EM_def}
\end{equation}
In the slender-loop approximation relevant for baryons, this contributes a leading $1/R$ piece to the mass functional, with a charge-dependent coefficient summarized in the baryon section as $\beta_Q/R$.

\noindent\emph{Idea:} track where source work goes. \;
\emph{Result:} it leaves as outgoing Poynting flux determined by the retarded fields.

\subsubsection{Tiny worked example: oscillating dipole (far zone)}
For a localized, time-harmonic electric dipole moment \( \mathbf p(t)=\mathbf p_0\cos\omega t\),
the leading far-field (\(r\gg\) source size) scales as
\begin{equation}
\big|\mathbf E(\mathbf r,t)\big| \sim \frac{\omega^2\,|\mathbf p_0|\,\sin\theta}{c^2\,r}\,
\cos\!\big(\omega(t-r/c)\big), \qquad
\big|\mathbf B(\mathbf r,t)\big| \sim \frac{|\mathbf E|}{c},
\end{equation}
and the radiated power scales like \(P \propto \omega^4 |\mathbf p_0|^2/c^3\).
(Exact angular factors and coefficients are standard and omitted here for brevity.)

\paragraph{Dictionary (for quick reading).}
\begin{itemize}
\item \(A^\mu\): electromagnetic 4-potential on the observed slice (EM only; no GEM mixing here).
\item \(F_{\mu\nu}\): field tensor; \(\mathbf E,\mathbf B\) are the measured electric and magnetic fields.
\item \(J^\mu_{\rm ch}=(c\rho_{\rm ch},\mathbf J_{\rm ch})\): charge/current; obeys \(\partial_\mu J^\mu_{\rm ch}=0\).
\item \emph{Lorenz gauge}: \(\partial_\mu A^\mu=0\); keeps \eqref{eq:wave_A_Lorenz} manifestly Lorentz-covariant.
\item \emph{Radiation condition}: only outgoing waves at infinity; selects retarded solutions.
\end{itemize}

\paragraph{Regime of validity.}
Equations in this wave-sector summary are understood at leading order in the EM small parameters defined in Sec.~\ref{sec:EM_validity}; see also Sec.~\ref{sec:motivation-conventions} for units and index/sign conventions. Subleading corrections enter at $\mathcal O(\varepsilon_\rho^2+\varepsilon_v^2+\varepsilon_\xi^2+\varepsilon_\kappa^2)$.
