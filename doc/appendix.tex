\appendix
\section{Nonlinear Scalar Field Equation}

This appendix provides a detailed derivation of the nonlinear extension of the scalar field equation, as used in the weak-field approximations throughout the main text. The equations are derived from the foundational postulates, particularly P-1 (compressible 4D medium with Gross-Pitaevskii dynamics) and P-3 (dual wave modes with density-dependent propagation). We focus on the irrotational sector for potential flow, assuming far-field neglect of quantum pressure and vector contributions; these can be reincorporated for core-scale or gravitomagnetic analyses. The derivation assumes a barotropic equation of state (EOS) projected from 4D, with effective speed $v_{\text{eff}}^2 = K \rho_{4D}$ where $K = g/m$.

Physically, this nonlinear equation governs unsteady compressible potential flow in the projected aether: time-varying potentials induce compression waves that propagate at variable speeds due to local rarefaction, while convective terms steepen inflows, potentially forming shock-like structures. Near aggregated vortex sinks (modeling massive bodies), density gradients slow $v_{\text{eff}}$, mimicking relativistic effects without invoking curvature.

\subsection{Projected Continuity Equation}

Begin with the 4D continuity equation from P-1:
\[
\partial_t \rho_{4D} + \nabla_4 \cdot (\rho_{4D} \mathbf{v}_4) = 0,
\]
incorporating vortex sinks from P-2 as localized drainage terms $-\sum_i \dot{M}_i \delta^4(\mathbf{r}_4 - \mathbf{r}_{4,i})$. Projecting to 3D (via integration over $w \sim \xi$, with $\rho_{3D} \approx \rho_{4D} \xi$ and aggregated sinks $\dot{M}_{\text{body}}$):
\[
\partial_t \rho_{3D} + \nabla \cdot (\rho_{3D} \mathbf{v}) = -\dot{M}_{\text{body}}(\mathbf{r}, t).
\]
For irrotational flow (P-4: $\mathbf{v} = -\nabla \Psi$):
\[
\partial_t \rho_{3D} - \nabla \cdot (\rho_{3D} \nabla \Psi) = -\dot{M}_{\text{body}}.
\]

\subsection{Projected Euler Equation}

The 4D Euler equation is:
\[
\partial_t \mathbf{v}_4 + (\mathbf{v}_4 \cdot \nabla_4) \mathbf{v}_4 = -\frac{1}{\rho_{4D}} \nabla_4 P - \frac{\dot{M}_{\text{body}} \mathbf{v}_4}{\rho_{4D}}.
\]
Projecting to 3D and assuming irrotationality ($\mathbf{a} = \partial_t \mathbf{v} = -\nabla \Psi$):
\[
-\partial_t \nabla \Psi + (\nabla \Psi \cdot \nabla) \nabla \Psi = -\frac{1}{\rho_{3D}} \nabla P + \frac{\dot{M}_{\text{body}} \nabla \Psi}{\rho_{3D}}.
\]
For barotropic EOS $P = (K/2) \rho_{4D}^2$ (projected as $P_{\text{eff}} \approx (K/2) (\rho_{3D}^2 / \xi^2)$), the pressure gradient integrates to enthalpy $h = \int dP / \rho_{4D} = K \rho_{4D}$.

\subsection{Streamline Integration and Bernoulli Form}

Integrate the Euler equation along streamlines (standard for potential barotropic flow):
\[
\partial_t \Psi + \frac{1}{2} (\nabla \Psi)^2 + K \rho_{4D} = F(t) + \int \frac{\dot{M}_{\text{body}}}{\rho_{3D}} \, ds,
\]
where $F(t)$ is a gauge function and the sink integral is localized near cores (neglected far-field for wave propagation). Gauging $F(t) = 0$:
\[
\rho_{4D} = -\frac{1}{K} \left( \partial_t \Psi + \frac{1}{2} (\nabla \Psi)^2 \right).
\]
(The negative sign ensures positive $\Psi$ yields deficits $\rho_{4D} < \rho_{4D}^0$.) With $\rho_{3D} \approx \rho_{4D} \xi$:
\[
\rho_{3D} = -\frac{\xi}{K} \left( \partial_t \Psi + \frac{1}{2} (\nabla \Psi)^2 \right).
\]

\subsection{Substitution into Continuity}

Substitute into the continuity equation:
\[
\partial_t \left[ -\frac{\xi}{K} \left( \partial_t \Psi + \frac{1}{2} (\nabla \Psi)^2 \right) \right] - \nabla \cdot \left[ -\frac{\xi}{K} \left( \partial_t \Psi + \frac{1}{2} (\nabla \Psi)^2 \right) \nabla \Psi \right] = -\dot{M}_{\text{body}}.
\]
Multiplying by $-K / \xi$:
\[
\partial_t \left( \partial_t \Psi + \frac{1}{2} (\nabla \Psi)^2 \right) + \nabla \cdot \left[ \left( \partial_t \Psi + \frac{1}{2} (\nabla \Psi)^2 \right) \nabla \Psi \right] = \frac{K}{\xi} \dot{M}_{\text{body}}.
\]
This quasilinear second-order PDE includes quadratic and cubic nonlinearities from convection and variable $v_{\text{eff}}$.

\subsection{Linear Regime Reduction}

In the linear limit ($\delta \Psi \ll 1$, $\rho_{3D} = \rho_0 + \delta \rho_{3D}$, $\delta \rho_{3D} = -(\rho_0 / c^2) \partial_t \delta \Psi$), calibrate $K / \xi = c^2 / \rho_0$ (far-field $v_{\text{eff}} = c$):
\[
\frac{1}{c^2} \partial_t^2 \Psi - \nabla^2 \Psi = 4\pi G \rho_{\text{body}},
\]
recovering the weak-field wave equation (Section 3.5), with $4\pi G \rho_{\text{body}} = (c^2 / \rho_0) \dot{M}_{\text{body}}$.

\subsection{Extensions and Applications}

\begin{itemize}
\item \textbf{Vector Coupling}: For frame-dragging, add solenoidal terms:
$\mathbf{a} = -\nabla \Psi + \partial_t (\nabla \times \mathbf{A}_g)$.
\item \textbf{Quantum Pressure}: Near cores, include $-\frac{\hbar^2}{2m \rho_{4D}} \nabla (\nabla^2 \sqrt{\rho_{4D}})$ in Euler for stability.
\item \textbf{Strong-Field Horizons}: Steady-state ($\partial_t \Psi = 0$) yields $|\nabla \Psi| = \sqrt{K \rho_{4D}}$ at ergospheres, calibrating to $r_s \approx 2GM/c^2$.
\item \textbf{Numerical Solves}: Finite differences can evolve $\Psi(t,\mathbf{r})$ for mergers or perturbations, predicting chromatic GW effects.
\end{itemize}

This nonlinear foundation distinguishes the model from GR through fluid-specific phenomena while recovering limits in weak fields.


\section{Golden-Ratio Fixed-Point Lemma}\label{app:phi-fixed-point}

Let $x>1$ denote a dimensionless pitch/twist ratio parametrizing braided configurations.
Define the involutive map $T:(1,\infty)\to(1,\infty)$ by $T(x)=1+1/x$ (``add one layer, then invert'').

\begin{lemma}[Exact invariance implies $\varphi$]
Suppose the coarse-grained energy $E:(1,\infty)\to\mathbb{R}$ is convex and admits a unique minimizer.
If $E\circ T = E$ exactly, then the unique minimizer satisfies $x_\star = T(x_\star)$ and hence $x_\star=\varphi=\frac{1+\sqrt{5}}{2}$.
\end{lemma}

\begin{proof}
If $E\circ T = E$ and $x_\star$ minimizes $E$, then $T(x_\star)$ is also a minimizer.
By uniqueness, $T(x_\star)=x_\star$, so $x_\star$ is a fixed point of $T$.
Solving $x=T(x)$ gives $x^2-x-1=0$, whose positive root is $\varphi$.
\end{proof}

\begin{corollary}[Approximate invariance gives a quantitative bound]
Assume $E$ is $m$-strongly convex on $(1,\infty)$ (i.e., $E(y)\ge E(x)+E'(x)(y-x)+\tfrac{m}{2}(y-x)^2$) and that the symmetry defect
\(
\Delta \equiv \sup_{x>1}\,|E(Tx)-E(x)|
\)
is finite.
Let $x_\star$ be the unique minimizer of $E$.
Then
\begin{equation}
|x_\star - \varphi| \;\le\; \sqrt{\tfrac{2\Delta}{m}}\,.
\end{equation}
\end{corollary}

\begin{proof}[Proof sketch]
By strong convexity and the definition of $\Delta$,
\(
E(Tx_\star) \ge E(x_\star) + \tfrac{m}{2}\,|T(x_\star)-x_\star|^2
\)
and
\(
E(Tx_\star) \le E(x_\star) + \Delta
\).
Hence $|T(x_\star)-x_\star| \le \sqrt{2\Delta/m}$.
Define $F(x)=T(x)-x$; then $F(\varphi)=0$ and $F'(x)= -1/x^2 - 1$, so $\inf_{x>1}|F'(x)|\ge 1$.
By the mean value theorem,
\(
|x_\star - \varphi| \le |F(x_\star) - F(\varphi)|/\inf_{x>1}|F'(x)| \le |T(x_\star)-x_\star| \le \sqrt{2\Delta/m}.
\)
\end{proof}

\noindent
In practice, $E$ is computed from tension, bending, and interaction terms under a constant-curvature/constant-torsion ansatz; convexity holds numerically across the parameter ranges explored, and $\Delta$ is small when twist--writhe trade-offs are nearly symmetric, matching the numerical observation $x_\star \approx \varphi$.

\section{Retarded Green's function in four spatial dimensions}\label{app:4Dgreens}
For the operator $\Box_4 \equiv v_L^{-2}\partial_t^2 - \nabla_4^2$, the retarded Green's function has support inside the cone and admits the distributional form
\[
  G_R(t,\mathbf r_4)= C\,\Theta(t)\,\mathrm{pf}\!\left[(v_L^2 t^2 - r_4^2)^{-3/2}\right]\Theta(v_L t - r_4),
\]
with normalization $C$ fixed by $\Box_4 G_R = \delta(t)\delta^{(4)}(\mathbf r_4)$. A brief derivation via Fourier transform and contour deformation is included here for completeness.

\section{Mollified projection: second-moment expansion and leading corrections}
\label{app:mollified}

We quantify how a finite transition width $\xi$ in the bulk direction $w$ modifies slice fields. Throughout, $\eta_\xi(w)=\xi^{-1}\eta(w/\xi)$ is an \emph{even}, smooth mollifier with unit mass $\int\eta=1$ and finite second moment
\[
\mu_2:=\int_{-\infty}^{\infty} s^2\,\eta(s)\,ds=O(1).\tag{D.1}
\]

\subsection{Projected kernel: $O\big((\xi/\rho)^2\big)$ control}
For the azimuthal kernel used in the circulation/grav sector,
\[
K_\rho(w)=\frac{\rho^2}{(\rho^2+w^2)^{3/2}},\qquad I(\rho)=\int_{-\infty}^{\infty}K_\rho(w)\,dw=2,\tag{D.2}
\]
the mollified integral is $I_\xi(\rho)=\int (\eta_\xi*K_\rho)(w)\,dw= \int K_\rho(w)\,dw$ by Fubini, so the \emph{value} is unchanged. What changes is any \emph{local sampling} of $K_\rho$ in $w$, which appears in intermediate steps. A standard even-moment Taylor estimate gives
\[
\big|(\eta_\xi*K_\rho)(w)-K_\rho(w)\big|
\le \frac{\mu_2\,\xi^2}{2}\,\big\|\partial_w^2K_\rho\big\|_{L^\infty(w-\delta,w+\delta)}
=O\!\big((\xi/\rho)^2\big),\tag{D.3}
\]
since $\partial_w^2K_\rho=O(\rho^{-2})$ for $|w|\lesssim \rho$. Consequently, any quantity built from $K_\rho$ and probed on in-plane scale $\ell\sim\rho$ inherits the same $O((\xi/\ell)^2)$ accuracy. This justifies the error terms used in the main text.

\subsection{Static potential: local closure and its small-$\xi$ form}
On the slice, the potential sector is closed by a linear, local operator acting on $\Phi$ and sourced by $\rho$,
\[
\mathcal{L}_\xi[\Phi]=\frac{\rho}{\varepsilon_0},\qquad
\mathcal{L}_\xi=-\nabla^2+\sum_{m\ge 2} a_{2m}\,\xi^{2m-2}\,\nabla^{2m},\tag{D.4}
\]
where even derivatives appear because $\eta_\xi$ is even. Truncating at the first nontrivial order gives the minimal model
\[
\big(-\nabla^2+\alpha\,\xi^2\nabla^4\big)\Phi=\frac{\rho}{\varepsilon_0},\qquad \alpha=O(1).\tag{D.5}
\]
In Fourier space ($\hat f(\mathbf k)$), this reads
\[
\hat\Phi(\mathbf k)=\frac{\hat\rho(\mathbf k)}{\varepsilon_0}\,\frac{1}{k^2\,(1+\alpha\xi^2 k^2)}.\tag{D.6}
\]
For a point source, $\hat\rho=q$, partial fractions yield
\[
\frac{1}{k^2(1+\alpha\xi^2 k^2)}=\frac{1}{k^2}-\frac{1}{1+\alpha\xi^2 k^2},\tag{D.7}
\]
and the inverse transform gives the Yukawa–regularized Green function
\[
\Phi(r)=\frac{q}{4\pi\varepsilon_0\,r}\Big(1-e^{-r/L}\Big),\qquad L:=\sqrt{\alpha}\,\xi.\tag{D.8}
\]
Thus: (i) the singularity is smoothed at $r\!\lesssim\!L$; (ii) for $r\!\gg\!L$, the correction is exponentially small, recovering Coulomb. Any polynomial-in-$\xi$ correction in the static far field must therefore arise from \emph{geometry-induced multipoles} (e.g., near boundaries), not from the local, isotropic closure itself.

\paragraph*{Remark (contact structure).} Expanding \eqref{eq:A.6} at small $k$,
\[
\hat\Phi=\frac{\hat\rho}{\varepsilon_0}\Big(\frac{1}{k^2}-\alpha\xi^2+O(k^2\xi^4)\Big),\tag{D.9}
\]
shows that beyond the Coulomb term, the leading analytic piece is $k$-independent and transforms to a contact (delta-like) contribution localized on sources. Away from sources, the static field remains Coulombic to this order.

\subsection{Waves: dispersion to leading order}
Allowing a finite exchange time $\tau$ in the displacement sector, the constitutive response in $(\omega,\mathbf k)$ takes the even form
\[
\varepsilon(\omega,\mathbf k)=\varepsilon_0\Big[1+\beta(\omega\tau)^2+\sigma (k\xi)^2+O\big((\omega\tau)^4,(k\xi)^4\big)\Big],\tag{D.10}
\]
with $\beta,\sigma=O(1)$. In vacuum ($\rho=\mathbf J=0$) the wave equation becomes
\[
k^2-\frac{\omega^2}{c^2}\,\Big[1+\beta(\omega\tau)^2+\sigma (k\xi)^2\Big]=0,\tag{D.11}
\]
so to leading order
\[
\omega^2=c^2 k^2\Big[1+\sigma (k\xi)^2+\beta (\omega\tau)^2\Big]
\;\Rightarrow\;
v_g=\frac{\partial\omega}{\partial k}=c\Big[1+\tfrac{3}{2}\sigma (k\xi)^2+\tfrac{1}{2}\beta (\omega\tau)^2\Big].\tag{D.12}
\]
This is the $\lambda^{-2}$ (spatial) and even-in-time $(\omega\tau)^2$ dispersion quoted in the EM section, preserving the homogeneous Maxwell identities exactly.

\subsection{Takeaway}
An even, thin transition profile produces \emph{quadratically suppressed} corrections controlled by $\xi$ (space) and $\tau$ (time). Statics: Coulomb is recovered outside sources, with near-field regularization at scale $L\sim\xi$ and exponentially small far-field deviations from the minimal local closure \eqref{eq:A.5}. Waves: the leading, falsifiable departures are the isotropic dispersions \eqref{eq:A.12}, scaling as $(k\xi)^2$ and $(\omega\tau)^2$.
